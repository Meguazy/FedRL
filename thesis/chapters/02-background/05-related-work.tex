\section{Related Work}

Our work draws on several areas of research: distributed reinforcement learning, federated learning applications to RL, behavioral diversity in multi-agent systems, and clustered federated learning.

\subsection{Distributed Reinforcement Learning}

Distributed training has become essential for reinforcement learning in complex domains due to the computational demands of both environment interaction and neural network training.

\textbf{Parallel Experience Collection:} Many RL systems use multiple actors to collect experience in parallel, dramatically increasing sample efficiency. A3C (Asynchronous Advantage Actor-Critic) introduced asynchronous updates from multiple workers to a shared model. IMPALA (Importance Weighted Actor-Learner Architecture) separates experience collection from learning, using importance sampling to handle the resulting off-policy data.

\textbf{Distributed AlphaZero:} The original AlphaZero training used distributed self-play, with many workers generating games in parallel while a central learner updates the neural network. This architecture enables the massive scale of training required for superhuman performance—AlphaZero played nearly 5 million games during training.

However, these distributed RL approaches still rely on centralized aggregation and aim for a single global model. They distribute computation for efficiency but do not address the challenge of maintaining behavioral diversity or training multiple specialized models collaboratively.

\subsection{Federated Reinforcement Learning}

Applying federated learning to reinforcement learning is an emerging research area. While traditional supervised federated learning deals with fixed datasets, federated RL must handle the added complexity of exploration, temporal dependencies, and non-stationary data distributions as policies improve.

\textbf{Policy-Based FedRL:} Some approaches extend FedAvg directly to policy gradient methods, aggregating policy network parameters across agents. However, this faces challenges when agents experience different environments or have different reward functions, as policies optimized for different MDPs may not meaningfully average.

\textbf{Value-Based FedRL:} Other work focuses on sharing value function estimates or Q-functions across agents. This can be effective when agents share the same environment but experience different parts of the state space.

\textbf{Exploration vs. Exploitation Trade-offs:} Federated RL introduces unique challenges for exploration. If all agents follow similar exploration strategies, they may collectively fail to explore the state space adequately. Some work addresses this through coordinated exploration strategies or by encouraging diversity in local training.

Most federated RL research has focused on settings where agents face different but related tasks, aiming to share knowledge across task distributions. Our work differs by considering agents working on the same task (chess) but seeking to maintain distinct behavioral strategies rather than converging to a single solution.

\subsection{Behavioral Diversity in Multi-Agent Systems}

Maintaining diversity in multi-agent systems has been studied in several contexts, motivated by applications in team behavior, robust learning, and ensemble methods.

\textbf{Quality Diversity Algorithms:} MAP-Elites and related algorithms explicitly optimize for both performance and behavioral diversity. They maintain archives of solutions that exhibit different behaviors, even if some are suboptimal, creating a diverse collection of strategies.

\textbf{Diversity-Driven Exploration:} In multi-agent RL, some work uses diversity objectives to encourage agents to explore different parts of the state space or learn different policies. This can improve collective exploration efficiency and robustness.

\textbf{Emergent Communication and Specialization:} Research in multi-agent communication has shown that agents can spontaneously develop specialized roles when working toward common goals, with different agents handling different subtasks.

Our work differs from these approaches in that we seek to maintain diversity not just during training but in the final models, and we do so in a federated setting where agents cannot directly observe each other but must coordinate through aggregation.

\subsection{Clustered Federated Learning}

Clustered federated learning recognizes that in heterogeneous settings, forcing all clients to converge to a single global model may be suboptimal. Instead, clients are grouped into clusters, with each cluster maintaining its own model.

\textbf{Automatic Clustering:} Several methods propose to automatically discover clusters during training. Clients are initially assigned to clusters randomly or based on data characteristics, then cluster membership is refined based on model similarity or gradient alignment. This allows the system to discover natural groupings in the data distribution.

\textbf{Multi-Center Federated Learning:} Some approaches maintain multiple global models and allow clients to contribute to the model that best matches their data. This creates a form of competitive federated learning where models specialize to different data distributions.

\textbf{Hierarchical Aggregation:} Similar to our approach, some work uses hierarchical aggregation where updates are first aggregated within clusters, then partial aggregation occurs across clusters. This balances the benefits of local specialization with global knowledge sharing.

However, existing clustered federated learning work focuses on supervised learning tasks with heterogeneous data distributions. The clustering emerges from data heterogeneity rather than being designed to preserve behavioral characteristics. Our work extends these ideas to reinforcement learning where we explicitly initialize and maintain clusters based on strategic playing style, and we introduce selective layer-wise aggregation to balance knowledge transfer with behavioral preservation.

\subsection{Transfer Learning in Deep RL}

Transfer learning in reinforcement learning aims to leverage knowledge from one task to accelerate learning on related tasks. This is relevant to our selective aggregation mechanism.

\textbf{Progressive Neural Networks:} Freeze previously learned networks and add new capacity for new tasks, allowing lateral connections. This prevents catastrophic forgetting but increases model size.

\textbf{Fine-Tuning and Layer Freezing:} Standard practice is to fine-tune pretrained networks on new tasks, often freezing early layers that learn general features while adapting later layers to task specifics.

\textbf{Multi-Task Learning:} Training a single network on multiple related tasks can improve performance on all tasks by learning shared representations. However, this typically assumes tasks are sufficiently similar that a shared representation helps rather than hinders.

Our selective inter-cluster aggregation draws on these insights. We share early feature extraction layers across playing style clusters, analogous to sharing general features in transfer learning, while keeping decision-making layers cluster-specific to preserve strategic differences.

\subsection{Gaps in Existing Work}

While the related work provides valuable insights and techniques, several gaps remain that our research addresses:

\begin{enumerate}
    \item \textbf{Federated RL with Intentional Diversity:} Existing federated RL work focuses on handling unavoidable data heterogeneity, not deliberately maintaining behavioral diversity as a goal.

    \item \textbf{Selective Layer Aggregation:} While some clustered FL work uses hierarchical aggregation, selective layer-wise aggregation based on functional role (feature extraction vs. decision making) is underexplored.

    \item \textbf{Playing Style Preservation:} Chess AI research has not addressed how to maintain distinct playing styles in collaborative training settings where the default would be homogenization.

    \item \textbf{Comprehensive Framework:} No existing work provides an end-to-end system combining clustered federated learning, selective aggregation, and self-play RL for behavioral diversity preservation.
\end{enumerate}

Our work addresses these gaps by developing a framework specifically designed to balance collaborative learning with behavioral preservation in reinforcement learning domains, using chess as a concrete and measurable testbed.
