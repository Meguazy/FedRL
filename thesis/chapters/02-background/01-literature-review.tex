\section{Literature Review Methodology}

To ensure a comprehensive and systematic review of relevant literature, we conducted a multi-stage search process across academic databases, preprint repositories, and technical documentation. This section details our search strategy, inclusion criteria, and the tools used to identify and synthesize relevant work.

\subsection{Search Strategy}

We performed systematic searches across multiple academic databases and repositories between September 2024 and January 2025. The primary sources included:

\begin{itemize}
    \item \textbf{Google Scholar:} Broad coverage of computer science literature and citation tracking
    \item \textbf{arXiv.org:} Recent preprints in machine learning (cs.LG, cs.AI, cs.MA)
    \item \textbf{ACM Digital Library:} Conference proceedings (NeurIPS, ICML, ICLR, AAAI)
    \item \textbf{IEEE Xplore:} Systems and distributed computing literature
    \item \textbf{Semantic Scholar:} AI-powered search and paper recommendations
\end{itemize}

\subsection{Search Terms and Queries}

Our literature search employed combinations of core terms, connected with Boolean operators. Table~\ref{tab:search-queries} shows the primary and secondary search queries used across different databases.

\begin{table}[h]
\centering
\caption{Literature Search Queries}
\label{tab:search-queries}
\begin{tabular}{|p{3cm}|p{10cm}|}
\hline
\textbf{Category} & \textbf{Search Query} \\
\hline
\multicolumn{2}{|l|}{\textbf{Primary Queries}} \\
\hline
Federated RL & "federated learning" AND "reinforcement learning" \\
\hline
Clustered FL & "clustered federated learning" OR "personalized federated learning" \\
\hline
Distributed Chess AI & "AlphaZero" AND ("federated" OR "distributed") \\
\hline
Behavioral Diversity & "behavioral diversity" AND "multi-agent" \\
\hline
Chess Playstyle & "chess AI" AND ("playing style" OR "playstyle") \\
\hline
Selective Aggregation & "selective aggregation" AND "federated learning" \\
\hline
Distributed MCTS & "Monte Carlo Tree Search" AND "distributed" \\
\hline
\multicolumn{2}{|l|}{\textbf{Secondary Queries}} \\
\hline
Heterogeneous FL & "heterogeneous federated learning" \\
\hline
Non-IID Data & "non-IID federated learning" \\
\hline
Transfer Learning & "transfer learning" AND "deep reinforcement learning" \\
\hline
Distributed Self-Play & "self-play" AND ("distributed" OR "federated") \\
\hline
Quality Diversity & "quality diversity algorithms" \\
\hline
Model Divergence & "model divergence" AND "federated" \\
\hline
\end{tabular}
\end{table}

We also performed backward citation tracking (reviewing references of key papers) and forward citation tracking (identifying papers that cite foundational work) to ensure coverage of relevant literature.

\subsection{Inclusion and Exclusion Criteria}

Papers were included if they met the following criteria:

\textbf{Inclusion criteria:}
\begin{itemize}
    \item Published between 2015 and 2025 (with exceptions for seminal earlier work)
    \item Directly relevant to federated learning, reinforcement learning, or chess AI
    \item Peer-reviewed or from reputable preprint repositories (arXiv)
    \item Available in English
    \item Sufficient technical detail to understand methodology
\end{itemize}

\textbf{Exclusion criteria:}
\begin{itemize}
    \item Purely theoretical work without implementation insights
    \item Domain-specific applications unrelated to game-playing or multi-agent systems
    \item Duplicate publications or superseded versions
    \item Insufficient detail on methods or results
\end{itemize}

\subsection{AI-Assisted Literature Discovery}

In addition to traditional database searches, we leveraged AI tools to assist with literature discovery and synthesis. Table~\ref{tab:ai-queries} shows the AI-assisted queries used for research assistance.

\begin{table}[h]
\centering
\caption{AI-Assisted Literature Discovery Queries}
\label{tab:ai-queries}
\begin{tabular}{|p{4cm}|p{9cm}|}
\hline
\textbf{Tool} & \textbf{Query / Purpose} \\
\hline
\multicolumn{2}{|l|}{\textbf{Claude (Anthropic)}} \\
\hline
Federated RL Overview & "Summarize recent advances in federated reinforcement learning, focusing on methods that handle heterogeneous agents" \\
\hline
Behavioral Diversity & "What are the key challenges in maintaining behavioral diversity in multi-agent systems?" \\
\hline
Selective Aggregation & "Compare different approaches to selective aggregation in federated learning" \\
\hline
Distributed AlphaZero & "Find papers that combine AlphaZero-style training with distributed or federated approaches" \\
\hline
\multicolumn{2}{|l|}{\textbf{Semantic Scholar}} \\
\hline
Recommendations & AI-powered paper recommendations based on citation graphs and content similarity \\
\hline
\multicolumn{2}{|l|}{\textbf{Connected Papers}} \\
\hline
Citation Networks & Visualizing citation networks and identifying research clusters \\
\hline
\end{tabular}
\end{table}

These AI-assisted searches were particularly useful for:
\begin{enumerate}
    \item Quickly understanding the landscape of a new research area
    \item Identifying terminology variations (e.g., "behavioral diversity" vs "policy diversity" vs "strategic heterogeneity")
    \item Discovering connections between seemingly disparate research communities (e.g., federated learning and chess AI)
    \item Generating additional search terms based on paper abstracts
\end{enumerate}

\subsection{Documentation and Synthesis}

We maintained a structured database of reviewed papers using reference management software, tracking:
\begin{itemize}
    \item Paper metadata (authors, venue, year)
    \item Key contributions and findings
    \item Methodological approaches
    \item Relevance to our research questions
    \item Gaps or limitations identified
\end{itemize}

This systematic approach ensured comprehensive coverage of relevant literature while maintaining focus on our core research questions about clustered federated learning for reinforcement learning with behavioral diversity preservation.
