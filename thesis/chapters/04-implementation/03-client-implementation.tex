\section{Client Implementation}

% This section describes the client-side (node) implementation.

\subsection{Federated Learning Node}

% Describe the main node class that manages client-side operations.
%
% Cover:
% - File: chess-federated-learning/client/node.py
% - Class: FederatedLearningNode
% - State machine: INITIALIZING → READY → TRAINING → UPDATING → IDLE → ERROR/SHUTDOWN
% - WebSocket client connection management
% - Async training task execution
% - Message handlers: _handle_start_training(), _handle_cluster_model()
% - Model serialization and transmission

\subsection{Trainer Implementations}

% Describe the pluggable trainer system and specific implementations.
%
% Cover:
% - File: client/trainer/trainer_interface.py
% - Abstract base: TrainerInterface with train() method
% - TrainingConfig and TrainingResult structures
% - Factory pattern: TrainerFactory for instantiation
%
% DummyTrainer (trainer_dummy.py):
%   - For testing without actual chess training
%   - Simulates training with random parameter modifications
%
% SupervisedTrainer (trainer_supervised.py):
%   - Loads Lichess PGN files (supports .pgn, .pgn.zst, .pgn.gz)
%   - Filters by playstyle using ECO codes
%   - Filters by rating (1800+ default)
%   - Creates ChessDataset for PyTorch DataLoader
%   - Training loop with combined policy + value loss
%
% PuzzleTrainer (trainer_puzzle.py):
%   - Loads Lichess puzzle database
%   - Filters by puzzle rating and themes
%   - Creates PuzzleDataset from FEN + move sequences
%   - Emphasis on tactical pattern learning

\subsection{Client Communication}

% Describe the WebSocket client implementation.
%
% Cover:
% - File: client/communication/client_socket.py
% - Class: FederatedLearningClient
% - Async message handling with callbacks
% - Auto-reconnect on connection loss
% - Parameter differencing for bandwidth reduction (sends Δθ instead of θ)
